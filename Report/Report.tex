\documentclass[11pt,a4paper]{article}

% --- PACKAGES ---
\usepackage[utf8]{inputenc}
\usepackage[margin=0.7in]{geometry}
\usepackage{graphicx}
\usepackage{hyperref}
\usepackage{titlesec}
\usepackage{enumitem}
\usepackage{amsmath}
\usepackage{xcolor}
\usepackage{libertine}
\usepackage[T1]{fontenc}
\usepackage{float}

% --- STYLING ---
\definecolor{headercolor}{RGB}{0,51,102}
\definecolor{linkcolor}{RGB}{0,102,204}

\hypersetup{colorlinks=true, linkcolor=linkcolor, urlcolor=linkcolor, citecolor=linkcolor}

\titleformat{\section}{\normalfont\large\bfseries\color{headercolor}}{\thesection}{1em}{}
\titleformat{\subsection}{\normalfont\normalsize\bfseries}{\thesubsection}{1em}{}
\titlespacing*{\section}{0pt}{8pt}{4pt}
\titlespacing*{\subsection}{0pt}{6pt}{3pt}

\setlength{\parindent}{0pt}
\setlength{\parskip}{0.32em}
\setlist{noitemsep, topsep=1pt, leftmargin=16pt}
\pagestyle{plain}

% Global equation spacing
\setlength{\abovedisplayskip}{3pt}
\setlength{\belowdisplayskip}{3pt}
\setlength{\abovedisplayshortskip}{2pt}
\setlength{\belowdisplayshortskip}{2pt}

\begin{document}

% ===== HEADER =====
{\Large\bfseries\color{headercolor} Bio-Inspired Path Integration with Intermittent Global Orientation Cues}

\vspace{0.25em}
{\normalsize\itshape A Simulation Study of Cataglyphis Navigation Applied to Infrastructure-Free Logistics}

\vspace{0.7em}
\begin{tabular}{@{}ll@{\hspace{1.8em}}ll}
\textbf{Name:} S Hari Preetham & \textbf{Email:} \href{mailto:haripreetham.jntuh@gmail.com}{haripreetham.jntuh@gmail.com} \\
\textbf{Phone:} +91 7013589964 & \textbf{Portfolio:} \href{https://hari-preetham-portfolio.vercel.app/}{hari-preetham-portfolio.vercel.app} \\
\multicolumn{2}{@{}l}{\textbf{Code:} \href{https://github.com/GODCREATOR333/Bio-inspired-Robotics}{github.com/GODCREATOR333/Bio-inspired-Robotics}}
\end{tabular}

\vspace{0.8em}

% ===== VIDEO UNDERSTANDING =====
\section*{1. Understanding of the Video \& Biological Model}

\textbf{Context:} The BBC Earth video showcases the desert ant \textit{Cataglyphis} navigating the Eastern Sahara—a featureless environment where temperatures reach 50°C, pheromone trails evaporate, and visual landmarks are nonexistent. The ant forages randomly for heat-exhausted prey but, upon finding food, returns in a \textbf{``dead straight line''} to its nest.

\textbf{Key Behavioral Observation:} The video explicitly shows the ant \textbf{``keeps stopping and making a turn.''} During these pirouettes, it checks the sun's position and polarized light pattern. I identified and codified three core behaviors:

\begin{enumerate}[noitemsep,topsep=0pt]
    \item \textbf{Path Integration (Dead Reckoning):} The ant measures distance between stops, continuously integrating noisy velocity vectors to maintain a mental ``home vector.''
    \item \textbf{Sun Compass (Global Orientation Cue):} The periodic ``stop and turn'' behavior where the ant samples global heading to correct accumulated drift.
    \item \textbf{Vector Homing:} Computing the ``shortest way home'' using the internal path integrator to navigate directly back to the nest.
\end{enumerate}

\textbf{Robotic Relevance:} This strategy directly addresses warehouse AGV navigation where geometrically uniform aisles cause Visual SLAM to fail (perceptual aliasing). Like the Sahara's featureless dunes, identical corridors lack distinctive landmarks. The ant's solution—combining noisy odometry with sparse global corrections—enables low-cost navigation without dense infrastructure (QR grids, guide rails). Ceiling beacons can serve as the ``sun,'' providing periodic orientation resets.

\section*{2. Codified Algorithm and Architecture}
This Python (PyQtGraph + OpenGL) framework uses a decoupled architecture to separate physical Ground Truth from the agent’s internal belief $\mathcal{N}(\hat{\mathbf{q}}, \mathbf{P})$, where $\hat{\mathbf{q}} = [\hat{x}, \hat{y}, \hat{\theta}]^\top$ to codify a Correlated Random Walk (CRW) for exploration and a Vector Homing policy for return, with the underlying state estimation maintained by a recursive EKF

\subsection*{A. Recursive State Estimation (EKF Prediction)}
The agent's uncertainity is quantified by a $3 \times 3$ covariance matrix $\mathbf{P}$. At each step of distance $d$ and turn $\delta\theta$, the belief is propagated through non-linear kinematics:
\begin{equation}
    \hat{\mathbf{q}}_{k+1} = f(\hat{\mathbf{q}}_k, \mathbf{u}) = \begin{bmatrix} \hat{x}_k + d\cos(\hat{\theta}_k + \delta\theta) \\ \hat{y}_k + d\sin(\hat{\theta}_k + \delta\theta) \\ \hat{\theta}_k + \delta\theta \end{bmatrix}, \quad \mathbf{P}_{k+1} = \mathbf{F}_x \mathbf{P}_k \mathbf{F}_x^\top + \mathbf{G} \mathbf{Q} \mathbf{G}^\top
\end{equation}
where $\mathbf{Q}$ is the process noise (stride/heading jitter) and $\mathbf{F}_x = \partial f / \partial \mathbf{q}$ is the Jacobian tracking how orientation error "sloshes" into positional drift. This results in the visual expansion of \textbf{95\% confidence ellipses} ($s=5.991$) along the movement axis.

\subsection*{B. Control Policy (Finite State Machine)}
The agent operates via five states: IDLE $\to$ SEARCH $\to$ FOUND\_FOOD $\to$ RETURN $\to$ STOP.
\begin{enumerate}[noitemsep, topsep=0pt, leftmargin=*]
    \item \textbf{SEARCH (Correlated Random Walk):} The physical path is generated by $\theta_t = \theta_{t-1} + \mathcal{N}(0, \sigma_{\text{turn}})$. This provides directional persistence for area coverage while the EKF simultaneously models the resulting integration drift.
    
    \item \textbf{RETURN (Vector Homing):} Upon food detection, the policy computes a homing vector $\vec{v} = -\hat{\mathbf{q}}_{1:2}$ using the \textit{estimated} EKF mean. The agent proceeds toward the perceived origin until $d_{\text{est}} < \tau$. If the true position $\mathbf{x}_{\text{true}}$ is outside the nest radius when $d_{\text{est}} \approx 0$, the EKF has diverged.
    
    \item \textbf{Correction Phase (EKF Update):} To bound error, a bio-mimetic sun compass provides an absolute heading $z = \theta_{\text{true}} + \mathcal{N}(0, \sigma_z)$. The estimator performs a measurement update:
    \begin{align*}
        \mathbf{K} &= \mathbf{P}\mathbf{H}^\top(\mathbf{H}\mathbf{P}\mathbf{H}^\top + R)^{-1} \quad &\text{(Kalman Gain)} \\
        \hat{\mathbf{q}} &\leftarrow \hat{\mathbf{q}} + \mathbf{K}(z - \mathbf{H}\hat{\mathbf{q}}), \quad \mathbf{P} \leftarrow (\mathbf{I} - \mathbf{K}\mathbf{H})\mathbf{P} \quad &\text{(Covariance Pinch)}
    \end{align*}
    where $\mathbf{H} = [0, 0, 1]$ isolates the heading state. This allothetic cue "pinches" the error ellipse, collapsing angular doubt and implicitly correcting correlated positional errors.
\end{enumerate}

% ===== RESULTS =====
\section*{3. Experimental Validation (Results averaged over $N=10$ independent trials per mode)}

\textbf{Mode 1—Blind (Estimated):} Relying purely on EKF prediction led to monotonic uncertainty growth. Once accumulated heading drift exceeded 90°, the belief-state homing vector became inverted, causing the agent to physically diverge from the origin while its internal model reported arrival at home.

\textbf{Mode 2—Control (True):} A zero-drift baseline where the agent utilized ground-truth coordinates for homing, resulting in perfect terminal accuracy.

\textbf{Mode 3—Sun Compass (EKF Update):} Periodic heading updates bounded orientation drift, which prevented the spatial covariance from exploding. This strategy achieved an 81\% reduction in mean terminal error compared to blind navigation, with a 90\% success rate in nest localization.

\begin{table}[H]
    \centering
    \small
    \renewcommand{\arraystretch}{1.1}
    \begin{tabular}{|l|l|c|c|}
    \hline
    \textbf{Mode} & \textbf{Strategy} & \textbf{Success Rate} & \textbf{Mean Terminal Error} \\ \hline
    1. Blind (Est) & EKF Prediction Only & 10\% & 52.00 mm \\ \hline
    2. Control & Ground Truth Baseline & 100\% & 2.96 mm \\ \hline
    3. Sun Compass & EKF Predict + Update & \textbf{90\%} & \textbf{9.72 mm} \\ \hline
    \end{tabular}
\end{table}

\begin{figure}[H]
    \centering
    \includegraphics[width=0.65\textwidth]{images/demo2.png}
    \caption{Simulation interface showing parameter controls (left), 3D View (top center), true vs. estimated position and error vs steps(bottom).}
    \label{fig:ui}
\end{figure}

\textbf{Video Demos:} 
Mode 1 (Blind—estimated): \href{https://youtu.be/hDBotoQ3_pA}{Watch} \quad | \quad 
Mode 2 (Control—true): \href{https://youtu.be/sY_AZQPQl3A}{Watch} \quad | \quad 
Mode 3 (Sun compass): \href{https://youtu.be/nh_rpq5Yy5c}{Watch}

\section*{4. Architecture and Research Scalability}
The system follows a modular, decoupled architecture, isolating the physical simulation from the estimator logic. To facilitate further study, the framework is wrapped as a \textbf{Gymnasium environment}, providing a standardized RL testbed. By exposing the EKF belief state as the observation space and normalized steering as the action space, the simulation acts as an API-driven platform for autonomous navigation research. This allows researchers to seamlessly deploy custom RL algorithms to optimize active sensing policies and investigate uncertainty-aware control in feature-sparse industrial environments.

\end{document}